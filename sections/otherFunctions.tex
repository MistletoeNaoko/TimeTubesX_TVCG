\section{Visual Exploration of Query Results}\label{sec:otherFunctions}
In addition to the feature and pattern detection methods described in Sections~\ref{sec:automaticExtraction} and \ref{sec:visualQuery}, TimeTubesX includes powerful visual comparison and annotation features for further analysis of blazar data.

\subsection{Visual Comparison of Query Results}
An essential feature for analysis of blazar datasets is the user's ability to compare query results---not just within a single query but also to previous query results. 
For example, when users find that a specific feature frequently appears in a certain time period,
they might want to investigate whether any other features also frequently appear in the same time period.
Therefore, TimeTubesX can juxtapose the results of different queries
by loading query results that were previously saved as a JSON file.
When importing a file, the stored results are mapped to a new timeline that is arranged as a juxtaposed view below the original timeline.
Hovering over marks on the timeline allows users to see detailed information about specific results.
Additionally, users can re-use or review the settings of the previous query, such as the selected time period or the variables assigned to the sketch pad (see Fig.~\ref{fig:UIFeatureExtraction}~(F)).

\subsection{Annotations of Queries and Query Results}
To enable efficient collaboration between astronomers and to facilitate keeping track of analyses between different sessions, TimeTubesX supports detailed annotations for query results.
Users can access any annotations even after exiting and restarting the application because annotations are stored in the local storage of the web browser. 
To share annotations with other users, annotations can be exported as a single JSON file. 
The system stores the annotation's timestamp, username, comment, and dataset, as well as detailed information about the query and query results.
Users can see all of their annotations in a table view or a single annotation by clicking on a marker with the selected label color in the TimeTubes and scatterplots views.
They can also re-use any query saved in an annotation by simply clicking on it. 

Annotations help users highlight interesting extraction results.
Annotating time intervals of interest not only triggers deeper inspection of a specific period 
but also possibly facilitates the discovery of new features such as periodic patterns.