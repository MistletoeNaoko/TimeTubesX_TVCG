\IEEEtitleabstractindextext{%
\begin{abstract}
Blazars are celestial bodies of high interest to astronomers. 
In particular, through the analysis of photometric and polarimetric observations of blazars, 
astronomers aim to understand the physics of the blazar’s relativistic jet. 
However, it is challenging to recognize correlations and time variations of the observed polarization, intensity, and color of the emitted light. In our prior study, we proposed TimeTubes to visualize a blazar dataset as a 3D volumetric tube. In this paper, we build primarily on the TimeTubes representation of blazar datasets to present a new visual analytics environment named TimeTubesX, 
into which we have integrated sophisticated feature and pattern detection techniques for effective location of observable and recurring time variation patterns in long-term, multi-dimensional datasets. 
Automatic feature extraction detects time intervals corresponding to well-known blazar behaviors. 
Dynamic visual querying allows users to search long-term observations for time intervals similar to a time interval of interest (query-by-example) or a sketch of temporal patterns (query-by-sketch). 
Users are also allowed to build up another visual query guided by the time interval of interest found in the previous process and refine the results. We demonstrate how TimeTubesX has been used successfully by domain experts for the detailed analysis of blazar datasets and report on the results.

% Blazars are celestial bodies of high interest to astronomers. In particular, astronomers aim to understand, through analysis of photometric and polarimetric observations of blazars, the physics of the blazar's relativistic jet. However, it is challenging to recognize time variation of and correlations among the observed polarization, intensity, and color of the emitted light. 
% We proposed TimeTubes, through which a blazar dataset is visualized as a 3D volumetric tube. In this paper, we develop TimeTubesX, a new visual analytics environment for blazar datasets that extends our previous TimeTubes. 
% In this paper, we describe TimeTubesX, a new visual analytics environment for blazar datasets that extends our previous TimeTubes, through which a blazar dataset is visualized as a 3D volumetric tube.
% We have integrated sophisticated feature and pattern detection techniques into TimeTubesX that allow for effective location of observable and recurring time variation patterns in long-term, multi-dimensional datasets. 
% Blazars are celestial bodies of high interest to astronomers. In particular, astronomers aim to understand, through analysis of photometric and polarimetric observations,
% the physics of the relativistic jet that is ejected from a blazar's center.
% However, it is challenging to recognize time variation patterns of and correlations among the observed polarization, intensity, and color of the emitted light. 
% We have previously proposed TimeTubes, through which a blazar dataset is visualized as a 3D volumetric tube.
% In this paper, we develop TimeTubesX, a new visual analytics environment for blazar observations that extends our previous TimeTubes.
% We have integrated sophisticated feature and pattern detection techniques into TimeTubesX that allow for the effective location of observable and recurring time variation patterns in long-term, multi-dimensional datasets.
% Automatic feature extraction methods detect time intervals that include well-known blazar behaviors. 
% Dynamic visual querying methods allow users to search long-term observations for time intervals similar to a time interval of interest (query-by-example) or a sketch of time variation patterns (query-by-sketch). 
% The result of a query can be directly re-used as an input for another dynamic visual query.
% This iterative pattern search allows users to build up another query guided by the interesting time interval found in the previous process and to refine results. 
% We demonstrate how TimeTubesX has been successfully used by domain experts for the detailed analysis of blazar observations and report on the results.
\end{abstract}

% Note that keywords are not normally used for peerreview papers.
\begin{IEEEkeywords}
Visual analytics, feature extraction, visual query, multi-dimensional, time-dependent visualization, astrophysics, blazar
\end{IEEEkeywords}}