\section{Related work\label{sec:relatedWork}}
Our target blazar datasets are multi-dimensional (mD), semi-structured, and time-dependent.
This section reviews prior work in astronomical visualization and visual query systems for mD and time-series data. 


\subsection{Visualization for Astronomical Data}\label{sec:relatedAstronomy}
Most astronomers work on analyzing signals from distant celestial bodies measured from the Earth~\cite{Kent2017}. 
ESASky~\cite{Baines2017} and OpenSpace~\cite{Bock2020} provide views for parts of the sky that are integrated from multiple data sources.
Almryde et al.~\cite{Almryde2016} visualize the temporal evolution of the position, mass, and radial velocity of dark matter halos. 
Li et al.~\cite{Li2008} realize simple visual analysis across multi-wavelength datasets.
IGM-Vis~\cite{Burchett2019} is a visual analytics tool for quasar sightline data.
Some approaches~\cite{Haroz2008, Li2008, Preston2016, McCurdy2019, Burchett2019} provide linked-view visualization systems that allow users to explore multiple types of visualizations for deeper analyses.
Many techniques have been developed for the analysis of astronomical data, 
but typically, blazar researchers have used animated scatterplots (see Section~\ref{sec:VisualEncoding}) 
because there were no other effective techniques for blazar observations prior to our early attempt, TimeTubes~\cite{Fujishiro2018}.

\subsection{Visual Queries}\label{sec:relatedFeature}
Visual query systems allow users to intuitively search for patterns of interest in their data.
Query-by-example aims to find data samples that are similar to a user-specified example,
whereas query-by-sketch allows users to directly draw the shape of interest.

\textsf{Query-by-example.\ }
Hochheiser and Shneiderman~\cite{Hochheiser2004} propose TimeSearcher, a visual exploration tool for time-series data. Users can select portions of timelines by using rectangular widgets~\cite{Buono2005, Buono2008}.
Holz and Feiner~\cite{Holz2009} propose a relaxed selection technique for time intervals through which users define the level of similarity either by the input speed or by spatial deviation from an original timeline.
The idea of detecting the input speed was adopted for our sketching interface.
However, these interactions are geared only toward univariate time-series data and are not appropriate for our mD blazar data.

Query-by-example methods for mD data have also been proposed.
Martin and Ward~\cite{Martin2005} describe brushing techniques for visualizations,
such as scatterplots matrices and parallel coordinate plots.
Elmqvist et al. propose interactions for constructing visual queries with star plots~\cite{Elmqvist2007} and multiple scatterplots~\cite{Elmqvist2008}.
Scribble query~\cite{Nielsen2016} allows users to form visual queries by scribbling axes of parallel coordinate plots.
Several works have focused on unstructured mD datasets~\cite{Martin2005,Elmqvist2007,Nielsen2016}, 
but they are not versatile enough to examine correlations among our idiosyncratic variables (see Section~\ref{sec:BlazarData}), and they do not support queries for structured time-series data.
The scatterplots by Martin and Ward~\cite{Martin2005} and Elmqvist et al.~\cite{Elmqvist2008} can be partially useful for structured data, but they cannot be applied to time-series data.

\textsf{Query-by-sketch.\ } 
Early examples of query-by-sketch tools include QuerySketch~\cite{Wattenberg2001} for temporal databases and QueryLines~\cite{Ryall2005} for multiple timelines. 
TimeSketch~\cite{Eichmann2015} supports pen- and touch-based query specification but focuses only on univariate time-series data.
To manage the ambiguities of hand-drawn sketches, several systems discretize the query and datasets into symbolic or quantized representations~\cite{Muthumanickam2016,ruta2019sax}.
Alternatively, Correll and Gleicher~\cite{Correll2016} define invariants for sketches. 
Qetch~\cite{Mannino2018} provides a matching algorithm that takes human sketching errors into account.
Our query-by-sketch interface uses sketching interactions similar to those of Qetch~\cite{Mannino2018}.
Both techniques target univariate time-series data.
Shao et al.~\cite{Shao2014} introduce sketch-based queries for large sets of scatterplots,
in which the system supports users with shadow-drawing suggestions. 
The suggestions allow users to draw a pattern based on actual values rather than their imaginations.
We use this idea as inspiration for our fact-guided querying (see Section~\ref{sec:factDrivenQuerying}).
While these systems support querying structured mD data, they cannot be applied to time-series data.

Several sketch-based techniques for multi-scalar volume data allow users to select regions of a 3D object through an ordinary 2D painting tool.
The techniques by Owada et al.~\cite{Owada2005} and Igarashi et al.~\cite{Igarashi2016} extract a plausible 3D region from the volume
by following contours or isosurfaces, respectively.
Tzeng et al.~\cite{Tzeng2003} present an method for specifying high-dimensional classification functions for the volume
by brushing regions of interest. 
BrainGazer~\cite{Bruckner2009} provides simple interactions for the selection of a region of interest in the volume
using drag-and-drop functionality.
These techniques can only target multi-scalar volume data. 

\textsf{Combinations of query-by-example and query-by-sketch.\ }
Query-by-example techniques do not address how to find the initial data samples of interest, 
while query-by-sketch techniques must surmount the user-introduced ambiguities of sketches.
Combining both styles helps to resolve this trade-off, as addressed by 
Bernard et al.~\cite{Bernard2010} and Lee et al.~\cite{Lee2019}. 
The system formulated by Bernard et al.~\cite{Bernard2010} supports univariate time-series data, while the one by Lee et al.~\cite{Lee2019} supports unstructured time-series data.
To get the best of both worlds, TimeTubesX supports both types of querying interactions.