\ifCLASSOPTIONcompsoc
\IEEEraisesectionheading{\section{Introduction}\label{sec:introduction}}
\else
\section{Introduction}
\label{sec:introduction}
\fi
\IEEEPARstart{B}{lazars} are important to the studies of astronomy and high-energy astrophysics. 
They are in a class of extremely bright galactic nuclei called active galactic nuclei (AGN) that shoot out a stream of particles from their center~\cite{Antonucci1993a}. 
A relativistic jet is angled directly toward the Earth from a central black hole of a blazar (see Fig.~\ref{fig:blazar}).
However, relatively little is known about the intricate physics of these jets' structures.
Astronomers pay careful attention to blazars
because the jet radiation of blazars is more amplified than that of other AGN.
To demystify the jets' structures,
astronomers need to analyze correlations and time variations in photometric and polarimetric observations of the emitted light.
For instance, a light burst (i.e., \textit{flare}) is one of the most distinctive behaviors of blazars~\cite{Bednarek1999, Atoyan2001}.
Identifying flares requires astronomers to scrutinize correlations of the emitted light's intensity with polarization and/or color, because looking at the time variation of the intensity alone is insufficient.
In addition to flares, some astronomers are interested in the polarization direction of light and whether it rotates~\cite{Marscher2008, Uemura2017}.
It is a controversial question in astronomy whether such \textit{rotations} are real or due to random variations of polarization.
To verify polarization rotations, correlations of the polarization with the intensity and/or the color are crucial.
Moreover, astronomers would like to locate time intervals with common features in  time variations
to understand what happens inside the jet.
However, it is an overwhelming process for them to manually examine all time intervals in long-term, multi-dimensional datasets in order to detect these significant patterns and validate hypotheses.

To support these exploration challenges, 
we have developed \textit{TimeTubesX}, a new visual analytics environment for blazar observations. 
This novel system extends our previous visualization scheme, called \textit{TimeTubes}~\cite{Fujishiro2018}, 
which allows users to interactively explore multi-dimensional, time-dependent observation datasets of blazars in a unique 3D tube view and to combine datasets from multiple observatories into a single visualization session termed \emph{visual data fusion}.
In TimeTubesX, we place more focus on visual analysis and automated and semi-automated approaches for feature and pattern detection.
The automatic feature extraction methods, which are in part an extension of our previous work~\cite{Sawada2018}, detect time intervals corresponding to well-known blazar behaviors, such as flares and rotations.
The dynamic visual querying methods, on the other hand, are useful for analysis and detection of recurring patterns that have not yet been explored in detail. 

Our first contribution is a detailed goal and task analysis performed with domain experts to design a visual analytics framework for blazar observations. 
Subsequently, we have designed and implemented TimeTubesX as a web-based and open-source system. 
Our second contribution lies in powerful automatic feature extraction methods for well-known blazar behaviors and an intuitive visual query platform for multi-dimensional, time-dependent data.
Finally, as our third contribution, we demonstrate the usefulness of TimeTubesX through experiments with a synthetic dataset, real data analyses, and feedback from domain experts.
We envision that our visual query mechanisms could be applied to multi-dimensional, time-series datasets in other domains to help users identify recurring patterns or similar time intervals.