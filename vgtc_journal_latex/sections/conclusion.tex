\section{Conclusion and future work\label{sec:conclusion}}
We have presented a web-based visual analytics environment for the detailed analysis of blazar datasets, termed TimeTubesX. 
TimeTubesX is expected to facilitate astronomers’ analysis of photometric and polarimetric behaviors of blazars 
by enabling automatic feature extraction and dynamic visual querying.
It allows astronomers not only to easily locate observable blazar behaviors but also to efficiently find recurring time variation patterns.
TimeTubesX has the potential to allow astronomers to find short time variation patterns that have not yet been discovered.

In the future, 
to avoid undesirable effects of outliers,
we should take observation errors into account during the feature extraction processes. 
We should also consider provenance management to holistically keep track of users’ analysis processes, as realized in aflak~\cite{Boussejra2019}.
Furthermore, we would like to incorporate a deep learning approach to classify time series data in a non-biased manner because the current query specification process in QBE and QBS significantly depends on users' expertise.
To provide a more effective overview of the results, it would also be helpful to cluster the results of a query. 
Applying TimeTubesX to multi-dimensional, time-dependent datasets in other domains will likely form yet another part of our future research.