\section{Conclusion and future work\label{sec:conclusion}}
We have presented a web-based visual analytics environment, termed TimeTubesX, for the detailed analysis of blazar datasets. %on top of our previous version of TimeTubes.
% Our system incorporates our prior work on 3D visualization of blazar data~\cite{Fujishiro2018} and further provides automatic feature extraction methods and dynamic visual queries that allow users to efficiently identify observable behaviors or recurring features from long-term datasets. 
% We introduced a novel query specification framework for time-dependent, multi-dimensional data. Users can refine a query iteratively to find interesting patterns through a fact-driven querying process.
TimeTubesX is expected to facilitate astronomers’ analysis of photometric and polarimetric behaviors of blazars 
by providing automatic feature extraction and dynamic visual querying.
% TimeTubesX has significantly improved astronomers' analysis of 
% photometric and polarimetric behaviors of blazars
% by providing automatic feature extraction and dynamic visual querying.
It allows astronomers not only to easily locate observable blazar behaviors but also to efficiently find recurring time variation patterns.
TimeTubesX has the potential to allow astronomers to notice short time variation patterns that have not yet been discovered.

% The feature and pattern detection in TimeTubesX still has some limitations.
% % First, our automatic feature extraction allows users to analyze long-term datasets and obtain candidates for characteristic behaviors, but our system cannot distinguish outliers from flares and it can miss rotations that include outliers.
% First, our automatic feature extraction cannot distinguish outliers from flares and it can miss rotations that include outliers.
% In the future, we should take the observation errors into account during the feature extraction processes to alleviate this problem. 
% % We currently do not take the observation errors into account during the feature extraction processes, which would alleviate this problem.
% Second, analyses based on user-driven dynamic visual querying can be biased because the query specification process in QBE and QBS significantly depends on users' experiences and expertise.
% Incorporating deep learning will be able to address this problem by classifying time series in non-biased manner.
% Finally, to give a more effective overview of the results, it would be helpful to cluster the results of a query, similar to the approach by Bernard et al.~\cite{Bernard2010}.

In the future, 
to distinguish outliers from flares and in order not to miss rotations that include outliers,
we should take the observation errors into account during the feature extraction processes. 
We also consider the provenance management to fully keep track of users’ analysis processes, as realized in aflak~\cite{Boussejra2019}.
Furthermore, we would like to incorporate a deep learning approach to classifying time series data in a non-biased manner because the current query specification process in QBE and QBS significantly depends on users' experiences and expertise.
To give a more effective overview of the results, it will also be helpful to cluster the results of a query. %, similar to the approach by Bernard et al.~\cite{Bernard2010}.
Applying TimeTubesX to multi-dimensional, time-dependent datasets in other domains will likely be yet another part of our future research.