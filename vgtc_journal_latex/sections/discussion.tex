\section{Discussion}\label{sec:discussion}
% feedback
% what was good, what worked well, what needs to be improved
% limitations of timetubes and future works
% We developed TimeTubesX with domain experts from Hiroshima University.
The domain experts from Hiroshima University not only performed the case studies in \autoref{sec:evaluation}, but also gave us detailed qualitative feedback.
In this section, we summarize their feedback and discuss current limitations of TimeTubesX.

%
Using TimeTubesX, our domain experts were able to find three rotation behaviors in the blazar \emph{3C 454.3}, which had not been described in literature yet. 
Whenever there are multiple polarization components in the same observation area, rotations cannot be identified just by analyzing $PA$ variations.
They said identifying these patterns was only possible because TimeTubesX allows them to analyze rotations whose centers are not at the origin at the Stokes plane~\cite{Huang2019}. 
% This is the case whenever there are multiple polarization components in the same observation area. In such a case, rotations cannot be identified by just analyzing the $PA$ variation. However, by applying our rotation detection method and using the visual queries of TimeTubesX, they were able to discover these hard to find rotation behaviors.

%
%When there are multiple polarization components in the observation area, the rotation center is not located at the origin of the Stokes plane. 
%In such a case, rotations cannot be identified just by scrutinizing $PA$ variation.
%Applying the rotation detection to a dataset for the blazar \emph{3C 454.3}, our collaborators found three rotation behaviors, which have not been recognized yet.
%They said TimeTubesX was helpful in addressing rotations whose center were not at the origin of the Stokes plane~\cite{Huang2019}.
%
The domain experts confirmed the usefulness of TimeTubesX. 
%generally gave positive comments regarding our dynamic visual queries. 
They found the visual queries helpful and impressive. % itself is interesting and quite impressive for them.
They said that especially the QBS method was useful for validating their abstract hypotheses, as demonstrated in \autoref{sec:evaluation}.
In addition, they said that TimeTubesX allowed them to identify interesting features in short time intervals.
This cannot be achieved with conventional methods because there are too many short time intervals in a long-term dataset.
Furthermore, fact-based querying was crucial for them because it allows them to refine their sketches based on actual extracted patterns. % make a sketch not just imaginary but practical based on real data.

The feature and pattern detection in TimeTubesX still has some limitations.
% TOO SPECIFIC
% Currently, it cannot extract time intervals with a mirrored shape of the input pattern.
% For example, to find out time intervals with a similar shape to the query in \autoref{fig:EvaluationQueryResults}~(a) but going in the opposite direction, users have to draw a new sketch. 
%
% Our domain experts also said that they would like to narrow down target time intervals from their extraction results. %, and to perform further feature extraction. 
% We do not support such narrowing retrieval yet.
% a progressive refinement functionality yet.
%
First, our automatic feature extraction allows users to analyze long-term datasets and obtain candidates for characteristic behaviors, but our system cannot distinguish outliers from flares and it can miss rotations that include outliers.
We currently do not take the observation errors into account during the feature extraction processes, which would alleviate this problem.
%
Second, analyses based on user-driven dynamic visual querying can be biased because the query specification process in QBE and QBS significantly depends on users' experiences and expertise.
% Incorporating statistical machine learning would be able to address this problem.
Incorporating deep learning would be able to address this problem by classifying time series in non-biased manner.
Finally, it would be helpful to cluster the results of a query, to give a more effective overview of the results, similar to the approach by Bernard et al.~\cite{Bernard2010}.
%which we currently do not support. Ideally, we would classify extraction results and thereby 
%Additionally, we plan to introduce a clustered result display,
%which classifies extraction results
%for the users to obtain an effective overview of the results, %like the approach by Bernard et al.~\cite{Bernard2010}.


% Regarding the automatic feature extraction, 
% the domain experts said 
% that the rotation detection could present time intervals which cannot be detected as rotations by the conventional methods used in astrophysical community.

% They said that it had a potential ability to spot time intervals satisfying the users' mental model from long-term observation.